\documentclass[a4j,uplatex,11pt]{jsarticle}

%%%%%%%%%%%%%%%%%%%%%%%%%%%%%%%%%%%%%%%%
% preamble
%%%%%%%%%%%%%%%%%%%%%%%%%%%%%%%%%%%%%%%%

% package
\usepackage{fancyhdr}
\usepackage{float}
\usepackage[dvipdfmx]{graphicx, hyperref}
\usepackage{pxjahyper}  % hyperref: link for Japanese href
\usepackage[top=20truemm,bottom=20truemm,left=20truemm,right=20truemm]{geometry}

% package settings
% - hyperref
\hypersetup{
	colorlinks=false, % リンクに色をつけない設定
	bookmarks=true, % 以下ブックマークに関する設定
	bookmarksnumbered=true,
	pdfborder={0 0 0},
	bookmarkstype=toc,
}

% macro
\newcommand{\linedhref}[2]{\underline{\href{#1}{#2}}}
\renewcommand\thefootnote{\arabic{footnote})}  % 脚注

% page style
\pagestyle{fancy}
\lhead{}

% title
\title{数学ガール 読書ノート}
\author{@pyteyon}
\date{2020/12/31 -}

%%%%%%%%%%%%%%%%%%%%%%%%%%%%%%%%%%%%%%%%
% document
%%%%%%%%%%%%%%%%%%%%%%%%%%%%%%%%%%%%%%%%

\begin{document}

\maketitle
\thispagestyle{empty}

% 目次
\tableofcontents
\newpage

% contents
\part*{はじめに}

\section{本ノートの目的}

本ノートの目的は筆者の数学力(という名の何か)の向上である.\\

筆者は数学が大の苦手である.受験では数 2B までしか使っておらず,かつ大学入試では 3 割取れれば良い方,というくらい数学が苦手である.
そのため,数学ガールを通して数学に対する感覚,考え方のベースを作りたいと考えている.\\

本ノートでは,数学ガールを読む際に浮かんだ疑問点,引っかかった点,覚えておきたいことなど,勉強を通じて考えたことを中心に記録していく.

\section{数学ガールとは}

数学ガールには以下二つのシリーズがある.

\begin{itemize}
	\item 「数学ガール」シリーズ
	\item 「数学ガールの秘密ノート」シリーズ
\end{itemize}

それぞれの概要は以下の通り.

\underline{「数学ガール」シリーズ}

\begin{quotation}
	「数学ガール」シリーズは,

	高校生の「僕」が数学ガールたちと数学に取り組む物語です.

	読み物形式でありながら,取り扱う数学的内容は本格的.

	ひとあじ違う数学にチャレンジしてみませんか.

	学生さんから社会人まで大人気!

	コミックスも英語版もあります.

	もちろん,電子書籍も!

	(引用元:\linedhref{https://www.hyuki.com/girl/}{著者の書籍説明の Web ページ})
\end{quotation}

\underline{「数学ガールの秘密ノート」シリーズ}

\begin{quotation}
	「数学ガールの秘密ノート」シリーズは,

	対話形式でやさしい数学に取り組む物語です.

	高校生の「僕」,高校生のミルカさんやテトラちゃん,

	中学生のユーリといっしょに数学に触れてみましょう.

	中学生・高校生が数学に親しむのにぴったり!

	どの巻からでも読めますよ.

	英語版もあります.

	もちろん,電子書籍も!

	(引用元:\linedhref{https://www.hyuki.com/girl/}{著者の書籍説明の Web ページ})
\end{quotation}

% 「数学ガール」シリーズ
\part{「数学ガール」シリーズ}

\section{数学ガール}
\section{数学ガール フェルマーの最終定理}
\section{数学ガール ゲーデルの不完全性定理}
\section{数学ガール 乱択アルゴリズム}
\section{数学ガール ガロア理論}
\section{数学ガール ポアンカレ予想}

\newpage

% 「数学ガールの秘密ノート」シリーズ
\part{「数学ガールの秘密ノート」シリーズ}

\section{数学ガールの秘密ノート 式とグラフ}
\section{数学ガールの秘密ノート 整数で遊ぼう}
\section{数学ガールの秘密ノート 丸い三角形}
\section{数学ガールの秘密ノート 数列の広場}
\section{数学ガールの秘密ノート 微分を追いかけて}
\section{数学ガールの秘密ノート ベクトルの真実}
\section{数学ガールの秘密ノート 場合の和}
\section{数学ガールの秘密ノート やさしい統計}
\section{数学ガールの秘密ノート 積分を見つめて}
\section{数学ガールの秘密ノート 行列が描くもの}
\section{数学ガールの秘密ノート ビットとバイナリー}
\section{数学ガールの秘密ノート 学ぶための対話}
\section{数学ガールの秘密ノート 複素数の広がり}
\section{数学ガールの秘密ノート 確率の冒険}

\newpage

% 参考文献
\bibliography{ref}
\bibliographystyle{junsrt}

\end{document}